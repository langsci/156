\documentclass[output=paper]{LSP/langsci} 
\author{Jürgen Pafel\affiliation{Universität Stuttgart} }
\title{Phrasal compounds and the morphology-syntax relation}
 
\abstract{Phrasal compounds are not an entirely uniform domain: it is necessary to distinguish between four different types of phrasal compounds. I will discuss their characteristics and the distinct analytical challenges. Only one type – the ›genuine‹ phrasal compounds with the non-head corresponding to a non-quotative well-formed syntactic phrase – poses a special problem for the morphology-syntax relation. There are three options for generating ›genuine‹ phrasal compounds: Merge, Insertion, and Conversion. I will argue that Conversion is the most suitable option. The analysis of phrasal compounds will suggest a symmetrical relation between word and phrase formation (phrases can be built on the basis of words  \textit{and} words on the basis of phrases) and a ›parallel‹ view of morphological and syntactic structure as fully separate structures with distinct properties.}
\ChapterDOI{10.5281/zenodo.896369}

\maketitle

\begin{document}
\section{Introduction} 
At first glance, phrasal compounds seem to be a phenomenon which
obviously demonstrates the intrusion of \isi{syntax} into morphology:
phrasal compounds seem to be words that contain syntactic phrases
({\ob}\textsubscript{N} XP - N{\cb}), i.e., phrasal compounds seem not
to obey Lexical Integrity. A thorough analysis of this phenomenon,
however, might suggest just the opposite: morphology and \isi{syntax} are
separate levels related by interface relations – so, at least, I will
argue.

With respect to the relation between morphological and syntactic structure, we can currently distinguish at least three different theoretical positions: morphological structure is a proper part of syntactic structure (Distributed Morphology); morphological and syntactic structures differ to a significant degree, but do overlap to some degree or interact \citep {AN04, LS06}; morphological and syntactic structures are fully separate structures with different properties (see, e.g., \citealt{Bresnan2001}, \citealt{Spencer2010}).
 
Related to these overall theoretical positions concerning the morphology-syn\-tax relation, we have three options for generating phrasal compounds, i.e., three options for relating an XP to the non-head of a compound – Merge, Insertion, and Conversion: we can form a phrasal compound either by merging an XP with the N head of the compound \citep {Lawrenz2006, LS06, Hein2015}, by inserting an XP in the non-head position of the compound \citep {AN04, Sato2010}, or by converting an XP into an N which functions as the non-head of the compound \citep {Harley2009, Pafel2015}.
 
We will approach these theoretical questions on the basis of a distinction between four different types of phrasal compounds which we can find in \ili{Afrikaans}, Dutch, \ili{English}, \ili{German}, Mandarin \ili{Chinese}, the \ili{Romance} languages, and \ili{Turkish}. What we generally call phrasal compounds are, as we will see, not an entirely uniform domain. I will present the characteristics of these four different types and show that they pose different challenges for analysis. There is just one type – the ›genuine‹ phrasal compounds with the non-head corresponding to a \isi{non-quotative} well-formed syntactic phrase – which poses a special problem for the morphology-\isi{syntax} relation, a problem which an account of phrasal compounds has to tackle. I will discuss the question of which of the options for generating phrasal compounds is appropriate to cope with genuine phrasal compounds in such a way that the relation to the other types of phrasal compounds is respected. I will argue that Conversion is the most suitable option (an option which relies on a certain input-output rule), and I will argue that such an account of phrasal compounds presupposes a clear distinction between two aspects of the morphology-\isi{syntax} relation: the relation between morphological and syntactic structure, on the one hand, and the relation between word formation and phrase formation, on the other. A thorough analysis of phrasal compounds suggests a symmetrical relation between word and phrase formation (phrases can be built on the basis of words \textit{and} words on the basis of phrases) and a ›parallel‹ view of morphological and syntactic structures as fully separate structures with distinct properties.
\section{Four types of phrasal compounds}

At first sight, one is inclined to define phrasal compounds as compounds whose non-head is a syntactic phrase, as is frequently done in the literature.\footnote{See, for instance, \citet [155] {Meibauer2003}, \citet [7] {Lawrenz2006}.} But, at closer inspection, it becomes evident that the examples discussed – for instance in the literature on \ili{German} phrasal compounds – are quite heterogeneous: not all of them strictly fit the initial definition. As for \ili{German}, we can distinguish between four types of ›phrasal compounds‹ which differ with respect to (i) the non-head (not) corresponding to a well-formed syntactic phrase [${\pm}$\textsc{well-formed}] and (ii) the non-head (not) being a quote [${\pm}$\textsc{quotative}].



\begin{table}
\begin{tabular}{lll}
\lsptoprule
 & +\textsc{well-formed} & –\textsc{well-formed}\\
\midrule
+\textsc{quotative} &  \bfseries Type I & \bfseries Type IV\\
\midrule
\scshape –quotative &
  \bfseries Type II &
 \bfseries Type III\\
\lspbottomrule
\end{tabular}
\caption{Types of phrasal compounds (\textsc{well-formed}=non-head being a well-formed syntactic phrase; \textsc{quotative}=non-head being a quote)}
\end{table}

Conceptually, the property of being a well-formed syntactic phrase is clear enough notwithstanding cases where it is difficult to decide whether a phrase is well-formed or not. The property of being quotative is more demanding. I use two criteria to distinguish quotative from \isi{non-quotative} phrasal compounds. Firstly, paraphrase with pure quotes: in contrast to \isi{non-quotative} phrasal compounds, the meaning of a quotative phrasal compound can most naturally be paraphrased using a \isi{pure quote} (e.g. Prince-of-Thieves film = film with the title `Prince of Thieves'). Secondly, interpretation of indexicals: in contrast to quotative phrasal compounds, indexicals in \isi{non-quotative} phrasal compounds are interpreted like ordinary indexicals with respect to the relevant utterance situation (compare below § 2.3).

As we will see, it is compounds of \textbf{Type II} – which we will call ›genuine phrasal compounds‹ – which pose a special problem for the morphology-\isi{syntax} relation, a problem which an account of phrasal compounds has to tackle. ›Quotative phrasal compounds‹ (i.e., \textbf{Type-I} and \textbf{Type-IV} compounds) do not pose a special problem for the morphology-\isi{syntax} relation as they are N(oun)N(oun) compounds as a consequence of having a quote as non-head, and neither do \textbf{Type-III} compounds (›pseudo-phrasal compounds‹) whose non-head does not correspond to a phrase at all. It will become evident that the different types pose distinct analytical challenges.

The classification in Table 1 seems to be cross-linguistically relevant. Languages other than \ili{German} seem to exhibit all four types (e.g., \ili{Afrikaans}, Dutch, \ili{English}, \ili{Turkish}) or at least some of them (e.g., Mandarin \ili{Chinese}, \ili{Romance} languages), and there are languages with no phrasal compounds at all (e.g., \ili{Polish} and other \ili{Slavic} languages). As we will see, the classification is compatible with results of diverse researchers investigating phrasal compounds in different languages.\footnote{What won't be dealt with here is to relate these types to the classification of semantic classes of heads, as we can find them in \citet [§6.1.1] {Meibauer2003}, \citet [§2.2] {TK15}, \citet [§2.3] {Goksel2015} and \citet [Kap. III.2.3] {Hein2015}.}

\subsection{Quotative phrasal compounds (\textbf{Type I})}

This type of phrasal compound ([+\textsc{well-formed}, +\textsc{quotative}]) consists of a noun preceded by a quote:
\ea
      \ea\label{ex:pafel:1a}
      \langinfo{Afrikaans}{}{\citealt[50; 57]{Savini1984}}\\
      \gll `hoe-gaan-dit-nog'-brief \\
             how-goes-it-well-letter \\
      \glt  `how-are-you letter’
      \ex\label{ex:pafel:1b}
      \langinfo{Afrikaans}{}{\citealt[50; 57]{Savini1984}}\\
      \gll  ek-het-nog-n'-kaart-in-die-mou-waarskuwing \\
             I-have-still-a-card-in-the-sleeve-warning \\
      \glt  `warning by someone that he has still a card up his sleeve’
    \z
\z

\ea
    \langinfo{Dutch}{}{\citealt[124]{AN04}, \citealt[148]{Booij2002}}\\
      \ea\label{ex:pafel:2a}  
        `waarom-leven wij?' probleem  \\                                          
      \glt  `why-do-we-live? problem' 
      \ex\label{ex:pafel:2b}
         Doe-het-zelf-winkel\\                                
      \glt  `Do-it-yourself shop'
      \ex\label{ex:pafel:2c}
         ver-van-mijn-bed-show\\                                
      \glt  `far-away-from-my-bed show'
    \z
\z

\ea
      \langinfo{English}{}{\citealt[324; 325; 326]{Trips2012}}\\
      \ea\label{ex:pafel:3a}      
        `wait and see' mentality\\
      \ex\label{ex:pafel:3b}
         `show the shirt' routine\\
      \ex\label{ex:pafel:3c}
         `kick me please' type\\
      \ex\label{ex:pafel:3d}
         Prince-of-Thieves film\\
    \z
\z

\ea
      \langinfo{German}{}{\citealt[45]{FB95}; \citealt[250]{Meibauer2007}}\\
      \ea\label{ex:pafel:4a}      
        Kaufe-Ihr-Auto-Kärtchen\\
            buy-your-car-card  \\
      \glt  ‘I-buy-your-car card’
      \ex\label{ex:pafel:4b}
         Lauf-dich-gesund-Bewegung \\             
      \glt  ‘run-yourself-fit movement'
      \ex\label{ex:pafel:4c}
         Trimm-dich-Pfad \\             
      \glt  ‘keep-fit path'
    \z
\z          

\ea
      \langinfo{Mandarin Chinese}{}{\citealt[185]{Wiese1996}, Fuyuan Zhou (personal communication)}\\
      \ea\label{ex:pafel:5a}      
      \gll  `yi-guo-liang-zhi'-zhengce\\
            one-country-two-system-politics\\
      \glt  `one-country-two-systems politics'
      \ex\label{ex:pafel:5b}
      \gll   `Bai-hua-qi-fang'-yundong\\
             hundred-flower-simultaneously-blossom-campaign\\
      \glt  `Hundred Flowers Campaign'
    \z
\z         

\ea
      \langinfo{Turkish}{}{\citealt[307; 308]{TK15}}\\
      \ea\label{ex:pafel:6a}      
      \gll  “tavuk-mu-yumurta-mı” soru-su \\
            chicken-\textsc{q}-egg-\textsc{q} question-\textsc{cm}\\
      \glt  ‘is-it-the-chicken-or-the-egg? question'
      \ex\label{ex:pafel:6b}
      \gll   “Bekle, gör-ür-üz” kafa-sı/tutum-u\\
             wait see-\textsc{aor}-1\textsc{pl} head-\textsc{cm}/attitude-\textsc{cm}\\
      \glt  ‘wait-and-(we shall) see-thinking/attitude’
    \z
\z     

The quote in these phrasal compounds is a ›\isi{pure quote}‹, not a ›citation‹ (cf. \citealt{Pafel2011} for this contrast). A \isi{pure quote} is part of a metalinguistic utterance as, e.g., in (\ref{ex:pafel:7}); a citation is part of a speech representation as, e.g., in (\ref{ex:pafel:8}). With respect to a citation, it makes sense to ask for the reference of indexicals and other referential expressions. Pure quotes differ: it makes no sense to ask for the reference of the \isi{indexical} in (\ref{ex:pafel:7}) – in contrast to (\ref{ex:pafel:8}):

\ea\label{ex:pafel:7}
      The sentence `I buy your car' is a declarative sentence. \\
\z
 
\ea\label{ex:pafel:8}
      She said to me: »I buy your car.« \\
\z 

The quotes in phrasal compounds behave like the \isi{pure quote} in (\ref{ex:pafel:7}): it makes no sense to ask for the reference of \textit{me} in (\ref{ex:pafel:3c}), \textit{Ihr} in (\ref{ex:pafel:4a}), \textit{dich} in (\ref{ex:pafel:4b}), or the persons alluded to by the \isi{suffix} \textit{üz} in (\ref{ex:pafel:6b}).

Research on quotation came independently to the conclusion that pure quotes are nouns (cf. \citealt[98 footnote 1]{Jespersen1924}; \citealt[Kap. III.2.2.1]{Klockow1980}; \citealt[153]{AN04}; \citealt{Pafel2007, Pafel2011}; \citealt[§5]{Vries2008}). Consequently, phrasal compounds of \textbf{Type I} are NN compounds and, semantically, they have the same structure as ordinary N\textsubscript{1}N\textsubscript{2} compounds: »being an N\textsubscript{2} which stands in relation R to N\textsubscript{1}« with R often being a pragmatically supplied relation of various kinds (as for the relation R in phrasal compounds compare \citealt{Meibauer2015}). See \REF{ex:pafel:9a} for illustration. The compound contains the quote \textit{`I buy your car'} and the head noun \textit{card}, and it has the meaning: »being a card displaying the writing `buy your car'« or, shorter, »card with the writing `buy your car'«


\ea
      \ea\label{ex:pafel:9a}
      \glt  \textit{Kaufe-Ihr-Auto-Kärtchen} = card with the writing `buy your car’\\ 
      \ex\label{ex:pafel:9b}
      \glt  \textit{Lauf-dich-gesund-Bewegung} = movement with the slogan ‘Run-yourself-fit’\\ 
      \ex\label{ex:pafel:9c}
      \glt  \textit{Prince-of-Thieves film} = film which has the title ‘Prince of Thieves’\\
     \z
\z

Multiple N \isi{recursion} is possible with phrasal compounds. \isi{Phrasal compounds of Type I} can be a proper part of compounds: they can be the head (see \ref{ex:pafel:91}) or the non-head of a compound (see \ref{ex:pafel:92}), and they even can be contained in a phrasal compound (in \ref{ex:pafel:93a} a phrasal compound of Type I is part of a phrasal compound of the same type, in \ref{ex:pafel:93b} it is part of a phrasal compound of Type III – cf. \ref{ex:pafel:15a}, and in \ref{ex:pafel:93c} it is part of a phrasal compound of Type II – cf. \ref{ex:pafel:25a}):

\ea\label{ex:pafel:91}
      \langinfo{German}{}{personal knowledge}\\
      \ea\label{ex:pafel:91a}      
      \gll  Pseudo-Trimm-dich-Pfad\\
            pseudo-keep-fit-path\\
      \ex\label{ex:pafel:91b}
      \gll   Hartz-IV-Trimm-dich-Pfad \\             
             Hartz-IV-keep-fit-path\\
      \ex\label{ex:pafel:91c}
      Hochglanz-`Kaufe Ihr Auto'-Kärtchen \\             
      high-gloss-`buy your car'-card\\
    \z
\z 
\newpage
\ea\label{ex:pafel:92}
      \langinfo{German}{}{personal knowledge}\\
      \ea\label{ex:pafel:92a}      
      \gll  Trimm-dich-Pfad-Gestaltung\\
            keep-fit-path-construction \\
      \ex\label{ex:pafel:92b}
      \gll   Trimm-dich-Pfad-Bewegung \\             
             keep-fit-path-movement \\
      \ex\label{ex:pafel:92c}
      \gll   `{Kaufe Ihr Auto}'-Kärtchen-Inflation \\             
             `{buy your car}'-card-inflation \\
    \z
\z 

\ea\label{ex:pafel:93}
      \langinfo{German}{}{personal knowledge}\\
      \ea\label{ex:pafel:93a}      
      \gll  `Du schaffst~es!'-Trimm-dich-Pfad\\
            `You succeed!'-keep-fit-path \\
      \ex\label{ex:pafel:93b}
      \gll   Vor-Trimm-dich-Pfad-Zeit \\             
             before-keep-fit-path-time \\
      \ex\label{ex:pafel:93c}
         Zwischen-den-Zeilen `Ihr~könnt~mich~mal'-Attitüde \\             
             between-the-lines-`Up~yours!'-attitude \\
    \z
\z 

Thus, phrasal compounds of \textbf{Type I} are regular NN compounds morphologically and semantically. Further they obey the principle »Words do not contain syntactic phrases«, i.e., they obey one version of Lexical Integrity (cf. \citealt{Pafel2015}). 

The fact that pure quotes are nouns has an interesting consequence. There must be some ›conversion‹ of phrases into words, as far as phrasal compounds of \textbf{Type I} are concerned. See the example in (\ref{ex:pafel:10}) and its analysis in (\ref{ex:pafel:11}) for illustration: the sentence \textit{I think so} is quoted, and is located at the position of the noun in a noun phrase, and it is inflected as a noun.

\ea\label{ex:pafel:10}
      \langinfo{English}{}{cf. \citealt[96 footnote 1]{Jespersen1924}}\\
        His speech abounded in many I think so’s. \\
\z 

\ea\label{ex:pafel:11}
      \ea\label{ex:pafel:11a}
      \glt {\ob}\textsubscript{sentence} I think so{\cb}  \\
      \ex\label{ex:pafel:11b}
      \glt {\ob}\textsubscript{noun phrase} many {\ob}\textsubscript{noun} I think so's{\cb}{\cb}  \\
      \ex\label{ex:pafel:11c}
      \glt {\ob}\textsubscript{word[N]} {\ob}\textsubscript{stem[N]} I think so{\cb} -s{\cb}  \\
     \z
\z

Thus, quotation and its analysis is a relevant topic, if we are interested in phrasal compounds. Note that possibly every language which exhibits phrasal compounds has phrasal compounds of \textbf{Type I}. Quotation is interesting as we find the same puzzling and challenging phenomenon: something which is a syntactic phrase gets a new life as a word or morpheme if it is quoted. Therefore, the question should be relevant to our topic of which options we have in dealing with generating pure quotes (see § 3).
 
Phrasal compounds of \textbf{Type I} are distinguished as a special class of phrasal compounds by several researchers partly independent of one another (see \citealt{Goksel2015}, \citealt{Pafel2015}, \citealt{TK15}).

As phrasal compounds of \textbf{Type I} are NN compounds, we could create a category of ›quotative compounds‹ as a special type of NN compounds: they either have a quote as non-head constituent (cf. \ref{ex:pafel:12} and the examples already presented of phrasal compounds of \textbf{Type I}), or they have a quote as the head of the compound (cf. \ref{ex:pafel:13}):

\ea\label{ex:pafel:12}
      \ea\label{ex:pafel:12a}
      \ili{English}{}{}\\
        \textit{for} phrases  \\
 
      \ex\label{ex:pafel:12b}
      \ili{German}{}{}\\
         \textit{für}-Phrasen    \\
      \glt  `\textit{for} phrases'
      \ex\label{ex:pafel:12c}
      \langinfo{Turkish}{}{\citealt[375]{Goksel2015}}\\
      \gll   yavasca sözcü-gü    \\
             slowly word-\textsc{cm}     \\
      \glt  `(the) word \textit{slowly}'
      \ex\label{ex:pafel:12d}
      \langinfo{Mandarin Chinese}{}{Fuyuan Zhou (personal communication)}\\
      \gll   \textit{ba}-zi-duanyu    \\
             \textit{ba}-sign-phrase     \\
      \glt  `\textit{ba}-phrase'
   \z
\z
 
\ea\label{ex:pafel:13}    
      \ili{German}{}{}\\
         Höflichkeits-\textit{Sie}    \\
      \glt  `politeness \textit{you}'
\z

Thus, in the end, what we called phrasal compounds of \textbf{Type I} can be subsumed under a subtype of NN compounds (cf. \citealt{Goksel2015}).
 
We can also deal with \textbf{Type IV} in the same vain. These phrasal compounds have a quote as non-head which is not a well-formed syntactic phrase, but a sequence of sentences or sentence-fragments. Compare the following examples (I made up examples (\ref{ex:pafel:14b}) and (\ref{ex:pafel:14c}) myself):\largerpage[2]

\ea
      \ea\label{ex:pafel:14a}
      \langinfo{German}{}{\citealt[142]{Schmidt2000}}\\
       `Versuche-mir-zu-verzeihen', `Ich werde-dich-ewig-lieben'-Briefchen   \\
            try-me-to-forgive, I-will-you-forever-love-letter\\
       `Try-to-forgive me', `I-will-love-you-forever letter'
      \ex\label{ex:pafel:14b}
      \langinfo{German}{}{personal knowledge}\\
         `Nein-vielleicht-doch-ja-vielleicht-aber-eigentlich-doch-nicht'-Gestammel\\
         no-perhaps-after-all-yes-perhaps-but-rather-after-all-not-stammering    \\
        `no-perhaps-after-all-yes-perhaps-but-rather-after-all-not stammering'\\
      \ex\label{ex:pafel:14c} \langinfo{English}{}{personal knowledge}\\
         `Hi-Hi-See-You' conversation\\
    \z
\z      


\subsection{Pseudo-phrasal compounds (\textbf{Type III})}

The non-head of these phrasal compounds ([–\textsc{well-formed}, –\textsc{quotative}]) neither corresponds to a well-formed syntactic phrase, nor is it quotative, compare \citet[139]{Lawrenz2006} and \citet{Pafel2015} for \ili{German}:

\ea
      \langinfo{German}{}{\citealt[44]{OM91}; \citealt[45]{FB95}; \citealt[146]{Schmidt2000}; \citealt[155]{Meibauer2003}}\\
      \ea\label{ex:pafel:15a}      
        Vor-Nobelpreis-Ära    \\
      \glt  `before-Nobel prize era'
      \ex\label{ex:pafel:15b}
        Vor-Ort-Bericht    \\
      \glt  `on-site report'
      \ex\label{ex:pafel:15c}
        Zweibettzimmer    \\
      \glt  `double bedroom'
      \ex\label{ex:pafel:15d}
        Vater-Sohn-Konflikt    \\
      \glt  `father-son conflict'
      \ex\label{ex:pafel:15e}
        Vorher-Nachher-Bilanz    \\
      \glt  `before-and-after account'
      \ex\label{ex:pafel:15f}
        Jeder-gegen-jeden-Krieg    \\
      \glt  `everyone-against-everyone war'
    \z
\z

The non-head constituent in (\ref{ex:pafel:15a}), i.e., \textit{Vor-Nobelpreis}, does not correspond to a well-formed syntactic phrase, but has a well-formed morphological structure which mimics a syntactic phrase in the sense that it is built by the same lexical material in the same order, exhibits a similar prosodic structure and is related to a phrasal semantics having the meaning »before the time when Nobel prizes were awarded«. 

\ea\label{ex:pafel:16}
        Vor-Nobelpreis-Ära   \\
      \glt  Morphological structure: {\ob}{\ob}P+N{\cb}\textsubscript{P} +N{\cb}\textsubscript{N}  \\
      \glt  Meaning: era before the time when Nobel prizes were awarded
\z

Therefore, it is not unreasonable to take them to be phrasal compounds, as the non-head exhibits properties of phrases, even if it does not correspond to a well-formed \textit{syntactic} phrase. The same holds for the non-heads in the other examples.
 
Phrasal compounds of \textbf{Type III} obey Lexical Integrity: the non-head con\-stit\-u\-ent of a pseudo-phrasal compound is not a well-formed syntactic phrase.
 
It seems that there are similar compounds in other languages, too – but it is at times difficult to judge whether or not the non-head corresponds to a well-formed syntactic phrase. 

\ea
\langinfo{English}{}{\citealt[323; 324]{Trips2012}}\\
      \ea\label{ex:pafel:17a}
        `famous for fifteen minutes' type   \\ 
      \ex\label{ex:pafel:17b}
         `first in last out' policy     \\
      \ex\label{ex:pafel:17c}
         `two for the price of one' sales    \\
      \ex\label{ex:pafel:17d}
         `always on the top' option 
    \z
\z

\ea
      \langinfo{Afrikaans}{}{\citealt[44; 65; 67; 71]{Savini1984}}\\
      \ea\label{ex:pafel:18a}      
      \gll  tafel-en-bank-eenheid\\
            table-and-bench-unit\\
      \glt  `unit consisting of a table and (a) bench'
      \ex\label{ex:pafel:18b}
      \gll   been-rek-ruimte \\
             leg-stretch-space\\
      \glt   `space in which to stretch one's legs'
      \ex\label{ex:pafel:18c}
      \gll   slaap-wakkerbly-patroon \\
             sleep-awake-stay-pattern\\
      \glt   `pattern of sleeping and staying awake alternately' 
      \ex\label{ex:pafel:18d}
      \gll   vaal-haar-nooi \\
             dull-hair-girl\\
      \glt   `girl with dull hair' 
      \ex\label{ex:pafel:18e}
      \gll   nege-oog-reus \\
             nine-eye-gaint\\
      \glt   `giant with nine eyes' 
   \z
\z
  
\ea
      \langinfo{Dutch}{}{\citealt[148; 150]{Booij2002}}\\
      \ea\label{ex:pafel:19a}      
        breed band antenne      \\
      \glt  `broadband aerial'
      \ex\label{ex:pafel:19b}
         twee persons bed \\
      \glt  `double bed'
     \ex\label{ex:pafel:19c}
         aardappel schrap machine \\
      \glt  `potato scraper'
     \ex\label{ex:pafel:19d}
         gooi-en-smijt-film \\
             throw-and-smash-film  \\
      \glt  `slapstick film'
    \z
\z
 
\ea
      \langinfo{Turkish}{}{\citealt[362]{Goksel2015}}\\
      \ea\label{ex:pafel:20a}      
      \gll  yan-ar dön-er meyva\\
            burn-\textsc{ptcp} turn-\textsc{ptcp} fruit\\
      \glt  Lit. `burning-turning fruit'
      \ex\label{ex:pafel:20b}
      \gll   ana baba gün-ü   \\
             mother father day-\textsc{cm}   \\
      \glt  `(a) crowded (place)'
    \z
\z          

The so-called \textit{polirematiche} `multiword expressions' in \ili{Romance} languages like the ones in (\ref{ex:pafel:21}) and (\ref{ex:pafel:211}) are sometimes called phrasal compounds. They consist of a noun followed by a preposition and a noun (N+P+N):

\ea\label{ex:pafel:21}
      \langinfo{Italian}{}{\citealt[397]{Bisetto2015}}\\
      \ea\label{ex:pafel:21a}      
        carta di credito    \\
      \glt  `credit card'
      \ex\label{ex:pafel:21b}
         unità di misura   \\
      \glt  `unit of measurement'
    \z
\z

\ea\label{ex:pafel:211}
      \langinfo{French}{}{\citealt[397]{Bisetto2015}}\\
      \ea\label{ex:pafel:211a}      
        verre à vin    \\
      \glt  `wine glass'
      \ex\label{ex:pafel:211b}
         fil de fer   \\
      \glt  `wire'
    \z
\z

According to \citet[397f.]{Bisetto2015}, the preposition and the following noun differ in their properties from PPs, and therefore it seems wrong to analyze the P+N part as a PP. This means they look like compounds of \textbf{Type III}.


\subsection{Genuine phrasal compounds (\textbf{Type II})}

The non-head of these phrasal compounds ([+\textsc{well-formed}, –\textsc{quotative}]) corresponds to a well-formed syntactic phrase, but it is not quotative.

\ea
      \langinfo{Afrikaans}{}{\citealt[39]{Savini1984}; \citealt[141; 142; 143]{Botha1980}}    \\
      \ea\label{ex:pafel:22a}      
      \gll  laat-in-die-aand drankie   \\
            late-in-the-evening drink   \\
      \glt  `drink taken late in the evening'
      \ex\label{ex:pafel:22b}
      \gll   uit-die-bottel-drink alkoholis  \\
             from-the-bottle-drink alcoholic  \\
      \glt  `alcoholic who drinks straight from the bottle'
      \ex\label{ex:pafel:22c}
      \gll   van-die-rak-pak  \\
             from-the-shelf-suit  \\
      \glt  `suit bought off the peg' 
      \ex\label{ex:pafel:22d}
      \gll   maklik-om-te-maak-poeding  \\
             easy-for-to-make-pudding  \\
      \glt  `pudding which is easy to make' 
   \z
\z
 
     
\ea
      \langinfo{Dutch}{}{\citealt[124]{AN04}; \citealt[146]{Booij2002}}\\
      \ea\label{ex:pafel:23a}      
        hoestend publick syndroom \\
      \glt  `coughing-audience syndrome'
      \ex\label{ex:pafel:23b}
         ijs met slagroom fobie  \\
      \glt  `ice-cream with whipped-cream phobia'
      \ex\label{ex:pafel:23c}
         vier-kleuren druk  \\
      \glt  `four-color printing'
      \ex\label{ex:pafel:23d}
         hete-lucht ballon \\
      \glt  `hot-air ballon'
    \z
\z      
   \largerpage[2]
\ea
      \langinfo{English}{}{\citealt[11]{Lieber1992}; \citealt[323]{Trips2012}}\\
      \ea\label{ex:pafel:24a}      
        over-the-fence gossip   \\
      \ex\label{ex:pafel:24b}
        slept-all-day look \\
      \ex\label{ex:pafel:24c}
        sex-in-shiny-packets literature 
    \z
\z
 
\ea
      \langinfo{German}{}{\citealt[161]{Brogyanyi1979}; \citealt[7]{Lawrenz2006}}\\
      \ea\label{ex:pafel:25a}      
        Zwischen-den-Zeilen-Widerstand   \\
      \glt  ‘between-the-lines resistance’
      \ex\label{ex:pafel:25b}
         In-Kontakt-bleiben-Geschenke   \\
      \glt  `keep-in-touch presents'
      \ex\label{ex:pafel:25c}
         Neid-auf-Reichtum-ohne-Leistung-Steuer   \\
      \glt  `envy-of-wealth-without-effort tax'
      \ex\label{ex:pafel:25d}
         Schwerer-als-Luft-Flugobjekte   \\
      \glt  `heavier-than-air flying objects'      
      \ex\label{ex:pafel:25e}
         Liebe-auf-den-ersten-Blick-Paar   \\
      \glt  `love-at-first-sight pair'
    \z
\z 

\ea
      \langinfo{Turkish}{}{\citealt[307; 308]{TK15}}\\
      \ea\label{ex:pafel:26a}      
      \gll  baba-lar ve ogul-lar toplanti-si  \\
            father-\textsc{pl} and son-\textsc{pl} meeting-\textsc{cm}    \\
      \glt  `fathers-and-sons meeting'
      \ex\label{ex:pafel:26b}
      \gll   tabiat-a dön-üs politika-si \\
             nature-\textsc{dat} return-\textsc{nom} policy-\textsc{cm}  \\
      \glt  `return-to-nature-policy'
      \ex\label{ex:pafel:26c}
      \gll   {\ob}Ne paha-sin-a olur-sa ol-sun tabiat-i kurtar-ma{\cb} politika-si \\
             what cost-3\textsc{sg}-\textsc{dat} be-\textsc{cond} be-\textsc{opt} nature-\textsc{acc} save-\textsc{nfnom} policy-\textsc{cm}  \\
      \glt  `Saving nature whatever the cost policy'
    \z
\z        
      
\ea
      \langinfo{Mandarin Chinese}{}{Fuyuan Zhou (personal communication)}\\
      \ea\label{ex:pafel:27a}      
      \gll  fan-fu-zhengce  \\
            against-corruption-policy  \\
      \ex\label{ex:pafel:27b}
      \gll   dusheng-zinü-zhengce   \\
             single-child-policy   \\
      \glt  `one-child policy'
    \z
\z 
         

The non-head constituent – for example \textit{over the fence} in (\ref{ex:pafel:24a}) – exhibits all characteristics of a well-formed phrase in form and meaning. The phrasal compound itself, however, has the canonical semantic structure of an N\textsubscript{1}N\textsubscript{2} compound: »being an N\textsubscript{2} which stands in relation R to N\textsubscript{1}« (being gossip which is transmitted over the fence). Or, see (\ref{ex:pafel:25a}): \textit{zwischen-den-Zeilen} `between the lines' is a well-formed PP and the compound has the meaning: »being a resistance which hides (or, is located) between the lines«
 
The exocentric VN compounds in \ili{Romance} languages like the ones in (\ref{ex:pafel:28}) marginally have a subtype where the verb combines with a phrase, an NP, as in (\ref{ex:pafel:29}).\footnote{Note that the compounds in (\ref{ex:pafel:29}) appear as the second noun in a superordinate compound in the corpus data of \citet{Bisetto2015}.}

\ea\label{ex:pafel:28}
      \langinfo{Italian}{}{\citealt[399f.]{Bisetto2015}}\\
      \ea\label{ex:pafel:28a}      
        cambiavalute  \\
      \glt  `money changer'
      \ex\label{ex:pafel:28b}
         portavalori   \\
      \glt  `amored car'  (lit. `carry valuables')
    \z
\z

\ea\label{ex:pafel:29}
      \langinfo{Italian}{}{\citealt[399f.]{Bisetto2015}}\\
      \ea\label{ex:pafel:29a}      
        ammazza {\ob}libertà digitali{\cb}  \\
      \glt  `digital freedom killing'
      \ex\label{ex:pafel:29b}
         ammazza {\ob}gente che non c'entra niente{\cb}   \\
      \glt  `killing people that have nothing to do with it'
    \z
\z

These compounds seem to belong to \textbf{Type II}. See \citet{Bisetto2015} for further candidates of phrasal compounds in \ili{Italian} (which we might classify as belonging to \textbf{Type II}).
 
\textbf{Type-II} compounds differ from quotative phrasal compounds (i.e., \textbf{Type-I} and \textbf{Type-IV} compounds) in the interpretation of indexicals (cf. \citealt[277]{Pafel2015}). We have seen in § 2.1 that it does not make sense to ask, with respect to \textbf{Type-I} compounds, for the reference of indexicals in the non-head. However, indexicals in the non-head of \textbf{Type-II} compounds differ. We can transform the attributive interrogative clause in (\ref{ex:pafel:30a}) into the non-head of a compound (\ref{ex:pafel:30b}) with no noticeable change of meaning (admittedly, (\ref{ex:pafel:30b}) is a quite uncommon way to say what the perfectly normal (\ref{ex:pafel:30a}) says – but it is a possible sentence):

\ea
      \langinfo{German}{}{personal knowledge}\\
      \ea\label{ex:pafel:30a}      
      \gll  Ich habe die Frage, ob ich glücklich bin, beantwortet.\\
            I have the question whether I happy am answered \\
      \glt  `I have answered the question of whether I am happy.'
      \ex\label{ex:pafel:30b}
      \gll   Ich habe die Ob-ich-glücklich-bin-Frage beantwortet.  \\
             I have the whether-I-happy-am-question answered   \\
      \glt  `I have answered the question of whether I am happy.'
    \z
\z      

The \isi{indexical} in the non-head in (\ref{ex:pafel:30b}) is interpreted in the same way as the \isi{indexical} being the subject of the sentence: they both refer to the speaker of the sentence. The fact that the \isi{indexical} in the non-head refers to the speaker of the sentence becomes even more evident when we modify the subject of the sentence: sentence (\ref{ex:pafel:31}) has the meaning that everyone answered the question of whether the speaker of (\ref{ex:pafel:31}) is happy, not the question of whether he himself is happy.

\ea\label{ex:pafel:31}
      \langinfo{German}{}{personal knowledge}\\
         Jeder hat die Ob-ich-glücklich-bin-Frage beantwortet.  \\
      \glt  `Everyone has answered the question of whether I am happy.'
\z

The relations change when we modify the compound into a quotative one. In this case, the indexicals are no longer interpreted with respect to the utterance situation of the sentence – note that (\ref{ex:pafel:32a}) and (\ref{ex:pafel:32b}) have the same meaning and cannot have the same meaning as (\ref{ex:pafel:31}):

\ea
      \langinfo{German}{}{personal knowledge}\\
      \ea\label{ex:pafel:32a}      
        Jeder hat die `Bin ich glücklich?'-Frage beantwortet. 
      \ex\label{ex:pafel:32b}
         Jeder hat die `Bist du glücklich?'-Frage beantwortet. \\
      \glt  `Everyone has answered the question of whether he himself is happy.'
    \z
\z      

Thus, we can use the interpretation of indexicals as a criterion to distinguish compounds of \textbf{Type I }\ and \textbf{Type II}. With this in mind, we find quite the same distinction in \ili{Turkish}: \citet{Goksel2015} distinguishes »quotational phrasal compounds« from »citational phrasal compounds«, and \citet[305]{TK15} distinguish between the »quotational« and the »nominalized«  type of phrasal compounds.  \largerpage[2]
It seems that compounds of this type are ›genuine‹ phrasal compounds, i.e., compounds with a true phrasal non-head: \isi{syntax}, semantics, and prosody point to this direction. Thus, they pose a challenge to the question of how to fit a phrase into a word.\footnote{Note that it is feasible to analyze the Dutch and \ili{German} compounds in (i) and (ii) as non-heads corresponding to a plural noun phrase containing a noun only (cf. \citealt[147]{Booij2002}).

\begin{enumerate}
 \item[(i)]  \textit{Dächermeer} (\ili{German}), \textit{dakenzee} (Dutch) `sea of roofs'
 \item[(ii)] \textit{Häuserreihe} (\ili{German}), \textit{hiuzenrij} (Dutch) `row of houses'
\end{enumerate}}
We will approach this question by having a look at how quotative phrases are fitted into a word. 



\section{Quotation and conversion}

In an article from 1984, Jackendoff came to the conclusion that »the phrase structure rule responsible for introducing {\ob}quotes{\cb} violates the normal theory of syntactic categories by permitting a totally free expression« (\citealt[26]{Jackendoff1984}). This consequence, however, is not mandatory. I know of two options dealing formally with pure quotes, both of which rely on conversion.
 
In their book \textit{Beyond Morphology}, Ackema and Neeleman take quoting to be zero-\isi{affixation}: »{\ob}T{\cb}he operation involves a change in syntactic status, both with respect to category and level of projection. Its input may be a syntactic phrase of any category, but its output consistently shows the distribution of a nominal head. {\ob}...{\cb} The formation of autoreferential expressions must hence be a case of zero \isi{affixation}« (\citealt[153-154]{AN04}).
 
Ackema and Neeleman further argue for an architecture where morphology and \isi{syntax} are distinct submodules of an encompassing module, generating distinct structures. Nevertheless, they tune their system in such a way that, under certain circumstances, merging of a syntactic phrase inside morphology is allowed. Zero-\isi{affixation} is a case in point.
 
Zero-\isi{affixation} to a syntactic phrase, however, is not sufficient to
deal with autoreferential expressions. Firstly, the phrases can be
fully ungrammatical, purely non-sensical, or they can mix different
languages. Secondly, not only phrases and words can be quoted, but
also morphemes, phonemes, graphemes. Pure quotes can thus not be built in Ackema/Neeleman's morphosyntactic module.\largerpage[1.5]
 
The alternative to zero-\isi{affixation} is conversion by an input-output rule which operates on expressions.\footnote{Note that I am not interested in the general controversy of whether or not conversion can be reasonably captured by zero-\isi{affixation}.} An expression can have several kinds of properties: \isi{phonological}, morphological, syntactic, semantic, and pragmatic ones. The rule takes an expression as input and gives another expression as output whose properties partially depend on the properties of the input expression. In the case of quoting a syntactic phrase, the rule takes an arbitrary expression (which is syntactically a phrase) as input and gives an expression as output which (i) surrounds the input expression's phoneme, or, better grapheme, sequence with quotation marks and which (ii) is morphologically a noun-stem. A decisive point of this input-output rule is that we can convert an expression with syntactic properties into an expression with morphological properties instead. This rule can be generalized as in (\ref{ex:pafel:33}) so that arbitrary linguistic elements can be converted into an expression which is morphologically a noun-stem (for details see \citealt{Pafel2015}).\footnote{The pure-quotation rule can easily take the form of an input-output rule which is formally of the same type as ›constructions‹ in the sense of \citet{Sag2012intro}, ›unary phrase structure rules‹ in \citet{Kay2014}, and ›lexical rules‹ in \citet{MW14}, which are all more or less on a par.}

\ea  Pure-quotation rule (simplified) \label{ex:pafel:33}\\
\begin{avm}
\[\scshape phon & \itshape phon \\
  ...
\]
\end{avm}
$\Rightarrow$
\begin{avm}
\[\scshape phon &  \itshape `phon' \\
  \scshape morph &  stem[N] \\
  \scshape sem & being of shape \itshape phon
\]
\end{avm}
\z

Phrasal compounds of \textbf{Type I} have a \isi{pure quote} as their non-head. That this quote is an N is the result of the application of the pure-quotation rule. Thus, constructing a phrasal compound of \textbf{Type I} is the concatenation of two nouns. The phrase-to-word conversion occurs ›previously‹ and is not part of the process of compounding. See for illustration the output of rule (\ref{ex:pafel:33}) for the quote in (\ref{ex:pafel:10}) \textit{His speech abounded in many }\textit{I think so}\textit{’s}:

%\todo{AS: Beispielnr.}
%(34)  Description of the \isi{pure quote} ‘I think so’\\
\ea Description of the \isi{pure quote} ‘I think so’\\
%%%  AS: dieses AVM machte Fehler
%  \begin{avm}
%   \[phon & <‘><I think so><’> >\\
%     morph & stem[N] >\\
%     being of shape <I think so>
%   \]
%  \end{avm}
%%%  AS: v2
\begin{avm} \label{ex:pafel:34}
 \[\scshape phon & <‘><I think so><’> \\
   \scshape morph & stem[N] \\
   \scshape sem & being of shape <I think so>
 \]
\end{avm}
\z

As for the morphology-\isi{syntax} relation, pure quotations show that words can be built in tandem with syntactic phrases, i.e., that phrases can be built on the basis of words \textit{and} words on the basis of phrases (phrase-to-word-conversion rules like the pure-quotation rule is the decisive element which makes it possible to build words on the basis of phrases). Nevertheless, we do not have to integrate morphology into \isi{syntax} to get this result. We can keep the morphological and the syntactic level apart from one another, as two separate dimensions of linguistic expressions.

\section{Three options of dealing with phrasal compounds}

There are, in principle, as I mentioned in the introduction, three options of generating phrasal compounds if the task is to solve the problem of how a phrase can be a base for a word. The options are Merge, Insertion, and Conversion: we can form a phrasal compound either by merging an XP with an N (the head of the compound), or by inserting an XP to the non-head position of the compound, or by converting an XP into an N which functions as the non-head of a compound. These options come with different accounts of the morphology-\isi{syntax} relation.
 
We know now that the different types of phrasal compounds require different analyses. Thus, it will not come as a surprise that these three options cannot account for all types of phrasal compounds. They are, first and foremost, options for dealing with genuine phrasal compounds (\textbf{Type II}), as we will see in a moment. Therefore, the question arises of how much these options differ from an adequate account of the other types. An option is preferred to the degree that it is related to the other accounts, i.e., an analysis of genuine phrasal compounds should not differ radically from the analysis of the other types.
 
The first option of dealing with genuine phrasal compounds is Merge. \citet{LS06} favor this option. They assume that there is a limited access of morphology to \isi{syntax}. Syntax and morphology have different principles in constructing phrases and complex words, respectively, and they »are normally blind to each other«. But for a limited domain, morphology can build complex words by merging syntactic phrases. The limited domain is determined in such a way that words with the structure {\ob}{\ob}XP{\cb} Y{\cb}\textsubscript{Y} become possible (cf. the very similar approach in \citet[§II.5]{Lawrenz2006} and the construction-grammar variant in \citet[42, 115]{Hein2015}). 
 
Merge seems adequate for genuine phrasal compounds (\textbf{Type II}), as their non-head constituent looks like a well-formed syntactic phrase (but note that we have a semantic interpretation which is typical for NN compounds (\citealt[141]{Lawrenz2006}), \citet{LS06} are silent on the semantic interpretation of Merge). This approach, however, is inadequate with respect to quotative phrasal compounds (\textbf{Type I}, \textbf{IV}) because they are NN compounds, as we have seen. Thus, a quite different approach would be necessary to cope with them, i.e., some kind of conversion. Merge is, also, inadequate for pseudo-phrasal compounds (\textbf{Type III}) whose non-head constituent is not a well-formed syntactic phrase. Summing up, the Merge approach plus an additional mechanism is best suited to account for morphology having access to \isi{syntax}, but it does not cover all types of phrasal compound and leads to a view of phrasal compounds where they appear to be a very heterogeneous set of phenomena.
 
\citet{AN04} have proposed to deal with phrasal compounds by a certain way of looking at the nature of insertion: insertion in their sense is just a way of feature matching. Morphology and \isi{syntax} differ substantially, but they are part of an encompassing module, and insertion allows for a limited interaction between them. A syntactic phrase (NP, for instance) can be inserted in an N slot of a NN compound as N and NP have matching features. In contrast to the Merge approach, Insertion takes phrasal compounds to be something which is made possible by the general way insertion works. 
 
However, categorial feature matching seems inadequate, as the
»inserted« XP can have various categorial features (nominal, verbal,
\isi{prepositional}, sentential etc.), which would predict that either the
phrasal compound can be of a type which is ruled out in some languages
(for instance, P(reposition)N(oun) compounds) or that is of a dubious
type (non-head corresponding to a sentence should be a word of which
category?), cf. \citet{LS06}.\footnote{Should it be the case that
  Ackema \& Neeleman take phrasal compounds to be always NN compounds,
  feature matching would become hollow
  (cf. \citealt[243]{Meibauer2007}; \citealt[392]{Sato2010}).}
Further, the following points speak against insertion. First,
Insertion does not cope with quotative phrasal compounds. We have
already seen that conversion is necessary to generate quotative
phrasal compounds. Ackema \& Neeleman would have to rely on
zero-\isi{affixation} to cope with them. Thus, quotative phrasal compounds
would differ in structure from genuine phrasal compounds: no XP is
inserted. Second, as for pseudo-phrasal compounds, the structure is
inadequate as the non-head constituent is not a syntactic
phrase. Ackema \& Neeleman's defending claim that the non-head constituent be a well-formed syntactic phrase in telegraphic speech is unconvincing. Take the phrasal compound (\ref{ex:pafel:35}) as an example. We could have (\ref{ex:pafel:36a}) as a headline, but not the unacceptable (\ref{ex:pafel:36b}).

\ea\label{ex:pafel:35}     
        Vor-Nobelpreis-Ära  \\
      \glt  ‘before-Nobel prize era’
\z

\ea
      \ea[]{\label{ex:pafel:36a}     
      \gll  Alles besser damals \\
            everything better then \\
      \glt  `Everything was better in former times.'
}
      \ex[*]{\label{ex:pafel:36b}
         Alles besser vor Nobelpreis  \\
      \glt  `Everything was better in the times before Nobel prizes were awarded.'
}
    \z
\z 

Insertion doesn't have to treat phrasal compounds as a peculiar phenomenon because the general process of insertion builds them under the assumption that there is limited interaction between morphology and \isi{syntax}. This predicts that we could find it in every language. But like Merge, it does not cover all types of phrasal compounds and leads to a view of phrasal compounds where they appear to be a quite heterogeneous set of phenomena. (For a similar approach with similar problems in a different framework see \citealt{Sato2010}.)
 
The Conversion approach proposes to deal with genuine phrasal
compounds by special phrase-to-word-conversion rules. According to
\citet{Harley2009}, a phrase undergoes zero-\isi{derivation} to a nominal
category, i.e., the complex phrase is affixed by a zero n head  (n\textsuperscript{0}): 

\ea\label{ex:pafel:37}     
      \glt  {\ob}{\ob}XP{\cb} n\textsuperscript{0}{\cb}\textsubscript{nP}       (where `nP' stands for `noun')
\z 

Harley endorses Distributed Morphology, but has to make quite »speculative« assumptions to integrate her analysis into this framework (note that even in a framework which treats word-formation purely syntactically, it is by no means easy to cope with phrasal compounds). As for semantics, the \isi{derivation} »will denote a concept evoked by the phrasal \isi{syntax}, though not compositionally determined by it« (\citealt[143]{Harley2009}); she further assumes that »quotative phrasal compounds evoke a particular attitude that might be attributed to a putative utterer of the phrase in question. Intuitively, the phrase has been fully interpreted, and an associated concept extracted from it — an attitude, in the case of quotatives, or an abstraction from an existing conceptual category, in the case of complex nP phrases as in \textit{stuff-blowing-up effects} or \textit{bikini-girls-in-trouble genre}« (\citealt[142]{Harley2009}). Apparently, Harley wants to account for quotative and genuine phrasal compounds syntactically and semantically in the same way, which neglects, however, the differences between these two types which we have presented.
 
According to \citet{Pafel2015}, a special input-output rule copes for genuine phrasal compounds. The rule in \REF{ex:pafel:pafel:xpton} takes a phrase (XP) as input and gives a noun as output. The phrase and the noun have exactly the same phonology and semantics, and the noun is a bound morpheme, as it does not occur outside of a nominal compound.

%(38) XP-to-N-conversion rule
\noindent\parbox{\textwidth}{\ea \label{ex:pafel:pafel:xpton} XP-to-N-conversion rule
\resizebox{\linewidth}{!}{\mbox{\begin{avm}
\[
 \scshape phon & \itshape phon \\
 \scshape syn & XP \\
 \scshape sem & predicate(x) \\
		& \itshape mean\]
\end{avm}
$\Rightarrow$
\begin{avm}
\[\scshape phon & \itshape phon \\
  \scshape morph & category: N  \\
          & valency: to-its-right-right(N) \\
  \scshape sem & predicate(x)\\
        & \itshape mean
\]
\end{avm}}}
\z}

Given an XP with an arbitrary \isi{phonological} form (\textit{phon}) and the semantics of a one-place predicate with an arbitrary meaning (\textit{mean}), the rule accounts for a word which has the same phonology as the phrase, as well as being of the morphological category N, selecting a noun to its right in morphology, and having the same semantics as the phrase. Note that SEM is a separate level for semantic structure, a level distinguished from syntactic structure (for arguments that it is, in coping with quantifier scope, necessary to distinguish syntactic and semantic level, see \citealt{Pafel2005}). As SEM but not SYN is relevant for semantic interpretation, the missing SYN feature in the output does not jeopardize semantic interpretation.
 
This operation can be seen as a kind of nominalization. Thus, we finally would get a canonical NN-compound structure for genuine phrasal compounds. Compare the nominalized gerund-like clauses as non-heads in \ili{Turkish} genuine phrasal compounds as discussed by \citet{TK15} and \citet{Goksel2015}. 

\ea\label{ex:pafel:39}
      \langinfo{Turkish}{}{\citealt[308]{TK15}}\\      
      \gll  {\ob}ic camasir-in-i göster-me{\cb} oyun-u  \\
            internal laundry-3\textsc{sg}-\textsc{acc} show-nfnom game-\textsc{cm}  \\
      \glt  `showing-your-underwear-game'
\z

Rule \REF{ex:pafel:pafel:xpton} is intended to capture genuine phrasal compounds only. Thus, there seems no progress with respect to Merge and Insertion. However, this time genuine and quotative phrasal compounds are captured by two variants of the same operation, i.e., phrase-to-word conversion. This captures the relation between the two phenomena. Additionally, there are two related morphological phenomena, namely phrasal \isi{derivation} and phrasal conversion, which ask for similar conversion analyses. In (\ref{ex:pafel:40}) a VP or NP is the base for the \ili{German} nominalizing suffixes \textit{{}-er}, -\textit{ung }or \textit{{}-artig}, in (\ref{ex:pafel:41}) a sentence is converted into a noun, a kind of exocentric word formation, and in (\ref{ex:pafel:42}) it depends on the details of analysis of whether this is a case of \isi{derivation} or conversion:

\ea\label{ex:pafel:40}
      \langinfo{German}{}{\citealt[8-9]{Lawrenz2006}}\\
      \ea\label{ex:pafel:40a}      
      \gll  Licht-in-Strom-Umwandl-er  \\
            light-in-current-convert-er \\
      \glt  `light-in-current converter'
      \ex\label{ex:pafel:40b}
      \gll   Kinder-über-Mittag-Betreu-ung\\
             children-on-noontime-caretake-ing  \\
      \glt  `children caretaking at noontime'
      \ex\label{ex:pafel:40c}
      \gll   Ruhe-vor-dem-Sturm-artig   \\
             quiet-before-the-storm-like  \\
      \glt  `like the quiet before the storm' 
   \z
\z
   \newpage
\ea\label{ex:pafel:41}
      \langinfo{German}{}{\citealt[9-10]{Lawrenz2006}}\\
      \ea\label{ex:pafel:41a}      
      \gll  (das) Wir-sind-wieder-Wer\\
            the we-are-again-someone  \\
      \glt  `(the general) attitude expressed by the slogan `We are somebody again''
      \ex\label{ex:pafel:41b}
      \gll   (das) Das-haben-wir-immer-schon-so-gemacht \\
             the this-have-we-always-already-so-done \\
      \glt  `(the) attitude express by the saying `We have done this ever since''
    \z
\z
   
\ea\label{ex:pafel:42}
      \langinfo{German}{}{\citealt[8]{Lawrenz2006}}\\      
      \gll  (das) Arm-um-die-Schulter-Legen  \\
            the arm-on-the-shoulder-put  \\
      \glt  `(the) resting of one's hand on someone's shoulder'
\z
      
As for the analysis of pseudo-phrasal compounds, we don't have to assume conversion, thus they differ from quotative and genuine phrasal compounds in this respect. They do, however, have the same structure insofar as they are XN compounds. Thus, there is only a minor difference to the structure of quotative and genuine phrasal compounds. 
 
In summary, much speaks in favour of the \isi{conversion approach}: it seems to deal with phrasal compounds in a satisfying manner, and it especially accounts for the relatedness of the four types of phrasal compounds without neglecting their differences.

\section{Conclusions}

Phrasal compounds are a challenge to the morphology-\isi{syntax} relation. The \isi{conversion approach} makes clear that we should distinguish between two aspects of this relation: the relation of morphological to syntactic structures, on the one hand, and the relation between word and phrase formation, on the other. As for the first aspect, the \isi{conversion approach} presented presupposes a parallel architecture where morphology and \isi{syntax} (and semantics) are separate structures (cf. \citealt{Bresnan2001}, \citealt{Spencer2010}, \citealt{Trips2016}). It is not necessary to modify the standard parallel relation between morphological and syntactic structure in order to cope with phrasal compounds. Lexical Integrity in the sense that (morphological) words do not contain phrases is fully respected (cf. \citealt{Pafel2015}). To the extent that the \isi{conversion approach} is successful, it contributes to the plausibility of a parallel architecture framework. As for the second aspect, phrasal compounds point to a symmetrical relation between word and phrase formation: phrases can be built on the basis of words \textit{and} words on the basis of phrases. This speaks against lexicalist approaches which claim that word formation strictly precedes the construction of syntactic phrases. Phrase-to-word-conversion rules (like \ref{ex:pafel:33} and \ref{ex:pafel:pafel:xpton}) is the decisive element which makes it possible to build words on the basis of phrases.
 
So we can conclude that phrasal compounds are only a phenomenon at first glance which suggests the intrusion of \isi{syntax} into morphology. A thorough analysis suggests just the opposite: morphology and \isi{syntax} are separate levels with fully separate structures with distinct properties.

This, then, means that, in morphology, we are dealing with (morphological) words, stems, affixes, etc., and in \isi{syntax}, we are dealing with (syntactic) words and phrases instead. The structures in morphology and \isi{syntax} are of quite different character. There is, however, some overlap with respect to the features assumed in morphology and in \isi{syntax}. Take the categorial and the gender feature as examples. In the default case, the morphological feature and its counterpart in \isi{syntax} are identical (a morphological noun, for instance, is a syntactic noun). An appropriate general interface relation copes for this identity. But there are interesting asymmetries, i.e., exceptions to this general interface relation. In \ili{German}, there is a class of words which, as far as \isi{syntax} is concerned, are undoubtedly nouns. But nevertheless they inflect like adjectives, exhibiting the strong/weak contrast and this is something that nouns normally never do. See the contrast in (\ref{ex:pafel:43}).

\ea\label{ex:pafel:43}
      \ili{German}{}{}\\
      \ea      
      \gll  ein fleißig-er Beamt-er\\
            a busy-\textsc{nom.m.sg.str} official-\textsc{nom.m.sg.str}     \\
      \glt  `a busy official'
      \ex
      \gll   der fleißig-e Beamt-e   \\
             the busy-\textsc{nom.m.sg.wea} offical-\textsc{nom.m.sg.wea}  \\
      \glt  `the busy official'
    \z
\z

We can account for this phenomenon if we distinguish morphological and syntactic categorial features. \citet{Spencer2010} has proposed analyzing these words syntactically as nouns and morphologically as adjectives.

Concerning gender, we also find an asymmetry. Take a look at the \ili{Latin} example \textit{agricola} `farmer'. It is syntactically masculine (as agreement suggests), but it is morphologically feminine (as \isi{inflection} suggests).


\ea\label{ex:pafel:44}
      \ili{Latin}{}{}\\
      \ea      
      \gll  sedul-us agricol-a\\
            busy-\textsc{nom.m.sg} farmer-\textsc{nom.f.sg}     \\
      \glt  `busy farmer'
      \ex
      \gll   sedul-i agricol-ae   \\
             busy-\textsc{nom.m.pl} farmer-\textsc{nom.f.pl}  \\
      \glt  `busy farmers'
    \z
\z      

In the default case, morphological and syntactic gender are identical,
of course. So like phrasal compounds, these asymmetries get a
straightforward analysis if morphology and \isi{syntax} are taken to be
separate levels related by interface relations.
\section*{Acknowledgements}
I am grateful for the stimulating discussions at the workshop im Mannheim in June 2015, as well as for Carola Trips' and an anonymous reviewer's comments on the manuscript.

\section*{Abbreviations}
\begin{multicols}{2}
\begin{tabbing}
\textsc{nfnom}\hspace{1em}\= non-factive nominalizer  \kill
\textsc{aor}  \> aorist        \\
\textsc{cm}   \> compound \isi{marker}    \\
\textsc{cond} \> conditional      \\
\textsc{nfnom}\> non-factive nominalizer  \\
\textsc{opt}  \> optative        \\
\textsc{ptcp} \> participle  \\
\textsc{str}  \> strong declension            \\
\textsc{wea}  \> weak declension\\
\end{tabbing}
\end{multicols}

{\sloppy
\printbibliography[heading=subbibliography,notkeyword=this]
}


 
\end{document}